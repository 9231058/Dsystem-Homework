\documentclass[paper=a4, fontsize=11pt]{article}

%----------------------------------------------------------------------------------------
%	PACKAGES AND OTHER DOCUMENT CONFIGURATIONS
%----------------------------------------------------------------------------------------

\usepackage{amsmath,amsfonts,amsthm} % Math packages
\usepackage{sectsty} % Allows customizing section commands
\allsectionsfont{\centering \normalfont\scshape} % Make all sections centered, the default font and small caps

\usepackage{fancyhdr} % Custom headers and footers
\pagestyle{fancyplain} % Makes all pages in the document conform to the custom headers and footers
\fancyhead{} % No page header - if you want one, create it in the same way as the footers below
\fancyfoot[L]{} % Empty left footer
\fancyfoot[C]{} % Empty center footer
\fancyfoot[R]{\thepage} % Page numbering for right footer
\renewcommand{\headrulewidth}{0pt} % Remove header underlines
\renewcommand{\footrulewidth}{0pt} % Remove footer underlines
\setlength{\headheight}{13.6pt} % Customize the height of the header

\numberwithin{equation}{section} % Number equations within sections (i.e. 1.1, 1.2, 2.1, 2.2 instead of 1, 2, 3, 4)
\numberwithin{figure}{section} % Number figures within sections (i.e. 1.1, 1.2, 2.1, 2.2 instead of 1, 2, 3, 4)
\numberwithin{table}{section} % Number tables within sections (i.e. 1.1, 1.2, 2.1, 2.2 instead of 1, 2, 3, 4)


\usepackage{listings}
\usepackage{graphicx}
\graphicspath{ {images/} }

\usepackage{xepersian}
\settextfont[Path=fonts/]{Vazir.ttf}
%\setlatintextfont{Times New Roman}

%----------------------------------------------------------------------------------------
%	TITLE SECTION
%----------------------------------------------------------------------------------------

\newcommand{\horrule}[1]{\rule{\linewidth}{#1}} % Create horizontal rule command with 1 argument of height

\title{
\normalfont\normalsize
\includegraphics[scale=0.1]{aut}
\hspace{5cm}
\includegraphics[scale=0.1]{ceit} \\
\textsc دانشگاه صنعتی امیرکبیر \\
\textsc دانشکده مهندسی کامپیوتر و فناوری اطلاعات
\horrule{0.5pt} \\ [0.4cm] % Thin top horizontal rule
\huge سیستم‌های توزیعی \\ % The assignment title
\huge پروژه سوم \\ % The assignment title
\horrule{2pt} \\ [0.5cm] % Thick bottom horizontal rule
}

\author{پرهام الوانی}

\date{\normalsize\today} % Today's date or a custom date

\begin{document}

\maketitle % Print the title

\paragraph{}
در فاز اول دو تابع \lr{map} و \lr{reduce}
با استفاده از تابع \lr{FieldsFunc} در زبان گو
که برای قطعه قطعه کردن رشته‌ها استفاده می‌شود پیاده‌سازی شدند.
همانطور که در صورت پروژه نیز شرح داده شده بود قرار بود رشته‌ی ورودی
با هر کاراکتر غیر از حروف قطعه قطعه شود و
به این منظور از تابع \lr{IsLetter} استفاده شد.

\paragraph{}
در فاز دوم یک سیستم توزیع‌شده برای انجام \lr{MapReduce}
در قالب یک کتابخانه پیاده‌سازی می‌گردد.

یک بخش از کد این کتابخانه در قالب \lr{master}
وظیفه دارد \lr{register} شدن
\lr{worker}ها
را سازمان‌دهی کند و کارهایی که در سیستم موجود است را به آن‌ها تخصیص دهد.

بعد از اتصال هر \lr{worker} یک \lr{go routine}
برای آن اجرا می‌گردد. این \lr{go routine}
در ابتدا تلاش می‌کند از کانال تسک‌های \lr{map}
تسک استخراج کرده و آن را به \lr{worker}
خود تخصیص دهد. در پایان هر تسک نتیجه‌ی آن
در کانال \lr{mapDone} منتشر می‌گردد.

بعد از اینکه \lr{master} مطمئن شد نتیجه‌ی تمام \lr{nMap}
را جمع آوری کرده است با بستن کانال \lr{reduceStart}
به تمام \lr{go routine}هایی
که \lr{worker}ها را مدیریت می‌کنند
اطلاع می‌دهد، می‌توانند وارد تسک‌های \lr{reduce} شوند.

در ادامه همان روند تسک‌های \lr{map} برای تسک‌های \lr{reduce}
طی شده و نتایج در کانال \lr{reduceDone} منتشر می‌گردد.

با پایان یافتن تسک‌های \lr{reduce} برنامه به صورت کامل خاتمه می‌یابد.
با توجه به روندی که شرح داده شد اضافه شدن \lr{worker}ها در میانه‌ی کار
امکان پذیر بوده و در صورت \lr{timeout} شدن هر \lr{worker}
تسک آن دوباره به کانال تسک‌ها اضافه شده و یک \lr{worker}
دیگر آن را برعهده خواهد گرفت.

\end{document}