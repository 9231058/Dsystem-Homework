\documentclass[paper=a4, fontsize=11pt]{article}

%----------------------------------------------------------------------------------------
%	PACKAGES AND OTHER DOCUMENT CONFIGURATIONS
%----------------------------------------------------------------------------------------

\usepackage{amsmath,amsfonts,amsthm} % Math packages
\usepackage{sectsty} % Allows customizing section commands
\allsectionsfont{\centering \normalfont\scshape} % Make all sections centered, the default font and small caps

\usepackage{fancyhdr} % Custom headers and footers
\pagestyle{fancyplain} % Makes all pages in the document conform to the custom headers and footers
\fancyhead{} % No page header - if you want one, create it in the same way as the footers below
\fancyfoot[L]{} % Empty left footer
\fancyfoot[C]{} % Empty center footer
\fancyfoot[R]{\thepage} % Page numbering for right footer
\renewcommand{\headrulewidth}{0pt} % Remove header underlines
\renewcommand{\footrulewidth}{0pt} % Remove footer underlines
\setlength{\headheight}{13.6pt} % Customize the height of the header

\numberwithin{equation}{section} % Number equations within sections (i.e. 1.1, 1.2, 2.1, 2.2 instead of 1, 2, 3, 4)
\numberwithin{figure}{section} % Number figures within sections (i.e. 1.1, 1.2, 2.1, 2.2 instead of 1, 2, 3, 4)
\numberwithin{table}{section} % Number tables within sections (i.e. 1.1, 1.2, 2.1, 2.2 instead of 1, 2, 3, 4)


\usepackage{graphicx}
\graphicspath{ {images/} }

\usepackage{xepersian}
\settextfont[Path=fonts/]{Vazir.ttf}
%\setlatintextfont{Times New Roman}

%----------------------------------------------------------------------------------------
%	TITLE SECTION
%----------------------------------------------------------------------------------------

\newcommand{\horrule}[1]{\rule{\linewidth}{#1}} % Create horizontal rule command with 1 argument of height

\title{
\normalfont\normalsize
\includegraphics[scale=0.1]{aut}
\hspace{5cm}
\includegraphics[scale=0.1]{ceit} \\
\textsc دانشگاه صنعتی امیرکبیر \\
\textsc دانشکده مهندسی کامپیوتر و فناوری اطلاعات
\horrule{0.5pt} \\ [0.4cm] % Thin top horizontal rule
\huge سیستم‌های توزیعی \\ % The assignment title
\huge فاز اول پروژه دوم \\ % The assignment title
\horrule{2pt} \\ [0.5cm] % Thick bottom horizontal rule
}

\author{پرهام الوانی}

\date{\normalsize\today} % Today's date or a custom date

\begin{document}

\maketitle % Print the title

\paragraph{}
برای پیاده‌سازی پروژه سه گام در نظر گرفته شده است. در اولین گام ارتباط بین سرور و کلاینت پیاده‌سازی شد
و در گام ارتباط بدون خطا فرض می‌شود.
در دومین و سومین گام به ترتیب ارتباط دارای خطا و بسته شدن صحیح سرور و کلاینت پیاده‌سازی گشت.

\paragraph{}
دو قسمت اصلی در کلاینت در نظر گرفته می‌شود، قسمت اول \lr{receiver}
است که داده‌ها را از سوکت دریافت کرده و در کانال \lr{incoming}
می‌ریزد. قسمت دوم \lr{handler}
نام دارد که در واقع قلب سیستم است و تمام کنترل‌ها در آنجا صورت می‌گیرد
دلیل این امر جلوگیری از وقوع \lr{race condition}
می‌باشد.

\paragraph{}
\lr{handler}
داده‌ها را از کانال \lr{incoming} دریافت می‌کند
و آن‌ها را در یک \lr{hash map} بر اساس شماره دنباله‌شان ذخیره می‌کند.
در ادامه پیام‌های \lr{ack} مناسب را تولید می‌کند.
پیام‌هایی که ترتیب آن‌ها درست است در کانال \lr{rmsg} برای دریافت توسط لایه‌ی بالاتر قرار می‌گیرند.

لایه اپلیکشن هم پیام‌هایی که می‌خواهد ارسال شوند را در کانال \lr{tmsg} قرار می‌دهد.
\lr{handler} این پیام‌ها را نیز در یک \lr{hash map} بر اساس شماره دنباله‌شان قرار می‌دهد
با این تفاوت که طول این \lr{hash map} هرگز نباید از \lr{window size (windows)}
بزرگتر شود.
با دریافت هر \lr{ack} آن پیام از لیست پیام‌های در حال ارسال حذف می‌شود.

\lr{handler}
با رویدادهای زمانی که در بازه‌های زمانی \lr{epochMillis} رخ می‌دهند
سه کار انجام می‌دهد. در صورتی که هیچ پیامی دریافت نشده باشد \lr{ack}
با شماره دنباله‌ی صفر ارسال می‌کند و اگر داده‌هایی در حال ارسال داشته باشد
تمام آن‌ها را ارسال می‌کند. در صورتی که هنوز ارتباط با سرور شکل نگرفته باشد پیام
\lr{connect}
را باز تکرار می‌کند.

برای بستن ارتباط یک کلاینت دو حالت وجود دارد. در اولین حالت
ارتباط با سرور \lr{timeout} می‌کند که در این حالت
ارتباط بدون توجه به وضعیت پیام‌ها بسته می‌شود و پیام خطا در
کانال \lr{err} قرار می‌گیرد.
در دومین حالت ارتباط به درخواست کاربر بسته می‌شود که در این حالت
نیاز است که \lr{handler}
تمام پیام‌هایی که در صف بافی مانده است را ارسال کند.
برای مدیریت این شرایط از متغیر \lr{status} استفاده می‌کنیم
\lr{status}
در واقع یک مانیتور است که دسترسی به یک \lr{integer}
را مدیریت می‌کند.

\paragraph{}
در سمت سرور هم از همان ایده‌ی کلاینت استفاده شده است فقط در اینجا
نیاز به دو \lr{handler}
داریم یکی برای سرور و دیگری به ازای هر کلاینت وجود خواهد داشت.
\lr{receiver} نیز مانند سابق پیام‌ها را دریافت کرده
و در کانال \lr{incoming} قرار می‌دهد.

رفتار \lr{handler} هر کلاینت مانند زمانی است که در کلاینت اجرا شده است
با این تفاوت که از یک کانال \lr{broadcast} برای مدیریت بستن هر کلاینت استفاده می‌شود.
کانال‌های \lr{broadcast} در go کانال خالی هستند که در زمان وقوع یک رویداد آن‌ها را می‌بندیم
در زمانی که یک کانال بسته می‌شود تمام کسانی که در حال دریافت از آن‌ها هستند مقدار صفر از آن کانال می‌خوانند،
به این ترتیب می‌توانیم تعداد زیادی \lr{goroutine} را از یک رویداد مطلع سازیم.

هر کلاینت می‌بایست بتواند خطا و داده‌ی خود را به لایه‌ی اپلیکشن منتقل کند. برای جلوگیری از
\lr{race condition}
کانال \lr{err} و \lr{rmsg}
به صورت \lr{global}
روی سرور تعریف شده‌اند و همه‌ی کلاینت‌ها داده‌های خود را به همراه شناسه‌شان در آن‌ها قرار می‌دهند.

\end{document}
