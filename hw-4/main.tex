\documentclass[paper=a4, fontsize=11pt]{article}

%----------------------------------------------------------------------------------------
%	PACKAGES AND OTHER DOCUMENT CONFIGURATIONS
%----------------------------------------------------------------------------------------

\usepackage{amsmath,amsfonts,amsthm} % Math packages
\usepackage{sectsty} % Allows customizing section commands
\allsectionsfont{\centering \normalfont\scshape} % Make all sections centered, the default font and small caps

\usepackage{fancyhdr} % Custom headers and footers
\pagestyle{fancyplain} % Makes all pages in the document conform to the custom headers and footers
\fancyhead{} % No page header - if you want one, create it in the same way as the footers below
\fancyfoot[L]{} % Empty left footer
\fancyfoot[C]{} % Empty center footer
\fancyfoot[R]{\thepage} % Page numbering for right footer
\renewcommand{\headrulewidth}{0pt} % Remove header underlines
\renewcommand{\footrulewidth}{0pt} % Remove footer underlines
\setlength{\headheight}{13.6pt} % Customize the height of the header

\numberwithin{equation}{section} % Number equations within sections (i.e. 1.1, 1.2, 2.1, 2.2 instead of 1, 2, 3, 4)
\numberwithin{figure}{section} % Number figures within sections (i.e. 1.1, 1.2, 2.1, 2.2 instead of 1, 2, 3, 4)
\numberwithin{table}{section} % Number tables within sections (i.e. 1.1, 1.2, 2.1, 2.2 instead of 1, 2, 3, 4)


\usepackage{graphicx}
\graphicspath{ {images/} }

\usepackage{xepersian}
\settextfont[Path=fonts/]{Vazir.ttf}
\setlatintextfont{Times New Roman}

%----------------------------------------------------------------------------------------
%	TITLE SECTION
%----------------------------------------------------------------------------------------

\newcommand{\horrule}[1]{\rule{\linewidth}{#1}} % Create horizontal rule command with 1 argument of height

\title{
\normalfont\normalsize
\includegraphics[scale=0.1]{aut}
\hspace{5cm}
\includegraphics[scale=0.1]{ceit} \\
\textsc دانشگاه صنعتی امیرکبیر \\
\textsc دانشکده مهندسی کامپیوتر و فناوری اطلاعات
\horrule{0.5pt} \\ [0.4cm] % Thin top horizontal rule
\huge سیستم‌های توزیعی \\ % The assignment title
\huge تمرین چهارم \\ % The assignment title
\horrule{2pt} \\ [0.5cm] % Thick bottom horizontal rule
}

\author{پرهام الوانی}

\date{\normalsize\today} % Today's date or a custom date

\begin{document}

\maketitle % Print the title

\section{سوال اول}
\paragraph{}
هر دوی این پروتکل‌ها به صورت مسدود کننده عمل می‌کنند، زمانی که آن‌ها می‌خواهند
عمل \lr{commit}
را انجام دهند می‌بایست منتظر بمانند تا همه‌ی شرکت کننده‌ها
\lr{commitVote}
را ارسال نمایند، به این ترتیب اگر یکی از شرکت کننده‌ها نتیجه را ارسال نکند سیستم تا ارسال نتیجه‌ی آن صبر می‌کند.
برای جلوگیری از این مشکل از \lr{timeout}
استفاده می‌کنند، به این ترتیب اگر یک پردازه تا زمان مشخصی جواب ندهد سیستم رای آن را
\lr{about}
در نظر می‌گیرد.

\section{سوال دوم}

\subsection{\lr{Centralized}}
\begin{itemize}
    \item تعداد پیام‌ها:
    \(3\)
    \item \lr{Request}, \lr{Ok}, \lr{Release}
    \item در صورتی که هماهنگ کننده‌ی اصلی از دست برود، تمام سیستم از کار می‌افتد.
    ولی در صورت خرابی گره‌ها سیستم می‌تواند به کار خود ادامه دهد.
\end{itemize}

\subsection{\lr{Token Ring}}
\begin{itemize}
    \item تعداد پیام‌ها
    \(1\) to \(\infty \)
    \item \lr{Token}
    \item در صورتی که \lr{token} از دست برود سیستم به صورت کامل از کار می‌افتد.
\end{itemize}

\subsection{\lr{Decentralized}}
\begin{itemize}
    \item تعداد پیام‌ها
    \[
    3 * m * k
    \]
    که در آن k تعداد تلاش‌ها جهت کسب اکثریت و m اندازه‌ی خوشه‌ی مربوط به منبع مورد نظر می‌باشد.
    \item \lr{Request}, \lr{Response}, \lr{Release}
    \item
    در این روش ممکن است \lr{starvation} رخ بدهد
    و یک پردازه هرگز نتواند به منابعی که می‌خواهد دسترسی پیدا کند.
    در ضمن این روش کارآیی پایین‌تری نسبت به سایر روش‌ها دارد ولی خطاپذیری آن از سایر روش‌ها بیشتر است.
\end{itemize}

\end{document}
